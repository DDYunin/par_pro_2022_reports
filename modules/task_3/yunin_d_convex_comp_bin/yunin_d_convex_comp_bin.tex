\documentclass{report}

\usepackage[warn]{mathtext}
\usepackage[T2A]{fontenc}
\usepackage[utf8]{luainputenc}
\usepackage[english, russian]{babel}
\usepackage[pdftex]{hyperref}
\usepackage{tempora}
\usepackage[12pt]{extsizes}
\usepackage{listings}
\usepackage{color}
\usepackage{geometry}
\usepackage{enumitem}
\usepackage{multirow}
\usepackage{graphicx}
\usepackage{indentfirst}
\usepackage{amsmath}

\geometry{a4paper,top=2cm,bottom=2cm,left=2.5cm,right=1.5cm}
\setlength{\parskip}{0.5cm}
\setlist{nolistsep, itemsep=0.3cm,parsep=0pt}

\usepackage{listings}
\lstset{language=C++,
        basicstyle=\footnotesize,
		keywordstyle=\color{blue}\ttfamily,
		stringstyle=\color{red}\ttfamily,
		commentstyle=\color{green}\ttfamily,
		morecomment=[l][\color{red}]{\#}, 
		tabsize=4,
		breaklines=true,
  		breakatwhitespace=true,
  		title=\lstname,       
}

\makeatletter
\renewcommand\@biblabel[1]{#1.\hfil}
\makeatother

\begin{document}

\begin{titlepage}

\begin{center}
Министерство науки и высшего образования Российской Федерации
\end{center}

\begin{center}
Федеральное государственное автономное образовательное учреждение высшего образования \\
Национальный исследовательский Нижегородский государственный университет им. Н.И. Лобачевского
\end{center}

\begin{center}
Институт информационных технологий, математики и механики
\end{center}

\vspace{4em}

\begin{center}
\textbf{\LargeОтчет по лабораторной работе} \\
\end{center}
\begin{center}
\textbf{\Large«Построение выпуклой оболочки для компонент бинарного изображения»} \\
\end{center}

\vspace{4em}

\newbox{\lbox}
\savebox{\lbox}{\hbox{text}}
\newlength{\maxl}
\setlength{\maxl}{\wd\lbox}
\hfill\parbox{7cm}{
\hspace*{5cm}\hspace*{-5cm}\textbf{Выполнил:} \\ студент группы 382006-2 \\ Юнин Д. Д.\\
\\
\hspace*{5cm}\hspace*{-5cm}\textbf{Проверил:}\\ доцент кафедры МОСТ, \\ кандидат технических наук \\ Сысоев А. В.\\\
}
\vspace{\fill}

\begin{center} Нижний Новгород \\ 2023 \end{center}

\end{titlepage}

\setcounter{page}{2}

% Содержание
\tableofcontents
\newpage

% Введение
\section*{Введение}
\addcontentsline{toc}{section}{Введение}
\par Множество различных задач вычислительной геометрии связано с построением выпуклой оболочки. Что же такое выпуклая оболочка? Выпуклой оболочкой множества X называется наименьшее выпуклое множество, содержащее X. В настоящий момент эта задача хорошо исследована и имеет широкое применение в распознавании образов, обработке изображений, а также в задачах в задаче раскроя и компоновки материала. В условиях данной лабораторной работы изображение считается заданным в оттенках серого. Во время обработки изображений часто требуется найти выпуклую оболочку, которая окружает объект на изображении.  Выпуклая оболочка похожа на полигональное приближение, за исключением того, что это выпуклый набор, который окружает самый внешний слой объекта. Этот выпуклый набор является пересечением всех выпуклых наборов, которые могут окружать объект.
\newpage

% Постановка задачи
\section*{Постановка задачи}
\addcontentsline{toc}{section}{Постановка задачи}
\par В рамках выполнения данной лабораторной работы требуется реализовать последовательный и параллельный (с использованием технологий MPI) алгоритмы построения минимальной выпуклой оболочки компонент бинарного изображения. Для проверки работы на ошибки необходимо реализовать ряд тестов с использованием Google Tests. Затем нужно провести вычислительный эксперименты для сравнения времени работы алгоритмов и сделать выводы об эффективности реализованных алгоритмов.
\newpage

% Описание алгоритма
\section*{Описание алгоритма}
\addcontentsline{toc}{section}{Описание алгоритма}
\par Задачу можно разделить на 3 глобальные подзадчи: определение компонент бинарного изображения, подготовка данных и построение выпуклой оболочки.
\paragraph{Определение компонент бинарного изображения\\\\}
\par Данный этап заключается в присвоение номера и определение каждой компоненты изображения. Это реализовано с помощью рекурсивного алгоритма для поиска компонент бинарного изображения.
\paragraph{Подготовка данных\\\\}
\par Целью этого этапа является уменьшения количества точек компоненты изображения и приведению их к нужному виду для алгоритма построения выпуклой оболочки. Поскольку изображение чаще всего имеет достаточно плотную структуру компоненты связанности, то многие точки могут быть исключены из рассмотрения, при построение выпуклой оболочки так как не могут быть ее вершиной. В этой лабораторной работе происходит удаление из рассмотрения точек, которые имеют более двух не нулевых соседей, также имеют двух не нулевых соседей на одной линии (справа и слева или сверху и снизу от текущего пикселя). Данное исключение поможет уменьшить количестве рассматриваемых точек до минимума, а кол-во оставшихся будет определятся сложностью фигуры компоненты связанности.
\paragraph{Построение выпуклой оболочки\\\\}
\par Для построения выпуклой оболочки для каждой компоненты был выбран алгоритм Грэхема. \\\\
\textbf{Алгоритм Грэхема}
\begin{enumerate}
  \item Находим точку, принадлежащую нашему множеству с самой левой x-координатой (если таких несколько, берем самую правую верхнюю из них), после добавляем в ответ.
  \item Сортируем все остальные точки по полярному углу относительно.
  \item Добавляем в ответ $p_1$ - самую первую из отсортированных точек. 
  \item Берем следующую по счету точку t. Пока t и две последние точки в текущей оболочке $p_i$ и $p_{i-1}$ образуют неправый поворот (вектора $p_i$$t$ и $p_i$$p_{i-1}$), удаляем из оболочки $p_i$. 
  \itemДобавляем в оболочку t. 
  \item Повторяем п.5, пока не закончатся точки.
\end{enumerate}
\newpage

% Описание схемы распараллеливания
\section*{Описание схемы распараллеливания}
\addcontentsline{toc}{section}{Описание схемы распараллеливания}
\par Как можно заметить, в последовательном алгоритме подготовка данных и алгоритм Грэхема происходят независимо для каждой компоненты. Значит каждому процессу назначается свой набор компонент (распределяется равномерно), для которых он определяет точки, которые образуют выпуклые оболочки. После чего данные, получившиеся в каждом процессе посылаются в 0 процесс, где они объединяются в один массив, в котором хранятся координаты точке выпуклых оболочек для каждой компоненты.
\newpage

% Описание программной реализации
\section*{Описание программной реализации}
\addcontentsline{toc}{section}{Описание программной реализации}
Программа содержит заголовочный файл convex\_comp\_bin.h и два файла исходного кода convex\_comp\_bin.cpp и main.cpp.
\par Заголовочный файл содержит также прототипы функций для вычисления выпуклой оболочки заданного набора точек, а также функции, среди которых: маркировка компонент изображения, вывода ветора точек, характеризует изображение, на экран, нахождения числа компонент, уменьшения количества точек и создания примеров векторов, которые характеризуют бинарное изображение.
\par Функция маркировки компонент изображения:
\begin{lstlisting}
vector<int> Labeling(const vector<vector<int>> &image, int width, int height);
\end{lstlisting}
Принимает на вход двумерный вектор - изображение, высоту и ширину этого вектора. Маркирует компоненты изображения и возвращает промаркированный вектор.
\par Функция вывода изображения на экран:
\begin{lstlisting}
void PrintImage(const vector<int> &image, int width, int height);
\end{lstlisting}
Принимает на вход вектор точек, его высоту и длину. Выводит на экран в удобном формате.
\par Функция инициализирования массива:
\begin{lstlisting}
void CreateLabelingImage(int image[], int width, int height);
\end{lstlisting}
Принимает на вход массив, длину и высоту. Заполняет его нулями.
\par Функция пометки точки:
\begin{lstlisting}
bool Fill(const vector<vector<int>> &image, vector<vector<int>> *label_image, int x, int y, int label, int height, int width);
\end{lstlisting}
Принимает на вход два вектора точек и 5 переменных типа int. Опредлеяет в какую точку нужно занести метку и определяет дальнейший шаг.
\par Функция алгоритма Грэхема:
\begin{lstlisting}
vector<int> GrahamAlgo(vector<int> points);
\end{lstlisting}
Принимает на вход вектор точек. Выполянет алгоритм на этом векторе и возвращает результат - вектор точек, входящих в выпуклую оболочку.
\par Функция, выполняющая задачу последовательно:
\begin{lstlisting}
vector<int> MainFuncSequence(const vector<int> &image, int width, int height, int num_components);
\end{lstlisting}
Принимает на вход вектор точек и три переменные типа int. В этой функции реализуется алгоритм решения задачи: помечаются компоненты, уменьшается число точек в каждой и строится выпуклая оболочка. В качестве результата выдаётся вектор точек.
\par Функция, выполняющая задачу параллельно :
\begin{lstlisting}
vector<int> MainFunParallel(const vector<int> &image, int width, int height, int num_components);
\end{lstlisting}
Принимает на вход вектор точек и три переменные типа int. В этой функции реализуется алгоритм решения задачи: помечаются компоненты, уменьшается число точек в каждой и строится выпуклая оболочка. В качестве результата выдаётся вектор точек.
\par Функция число точек компоненты:
\begin{lstlisting}
int CountNumPointsComponent(vector<int> image);
\end{lstlisting}
Принимает на вход вектор точек. Определяет сколько точек находятся в компоненте.
\par Функция убирает лишние точки компоненты:
\begin{lstlisting}
vector<int> MakeMinPointsLocal(const vector<int> &image, int width, int height, int component);
\end{lstlisting}
Принимает на вход вектор точек и три перемееные типа int. Выполянет алгоритм, по результатам котрого число точек в компоненте уменьшается до минимума, что позволяет быстрее строить выпуклую оболочку.
\par В файле исходного кода convex\_comp\_bin.cpp содержится реализация функций, объявленных в заголовочном файле. В файле исходного кода main.cpp содержатся тесты для проверки корректности программы.
\newpage

% Подтверждение корректности
\section*{Подтверждение корректности}
\addcontentsline{toc}{section}{Подтверждение корректности}
Для подтверждения корректности работы данной программы с помощью Google Tests мной было разработано 5 тестов: каждый из них, так или иначе, проверяет работоспособность программы при определенных, изначально заданных условиях. Проверяется как последовательная реализация, так и параллельная. Условием для подтверждения корректности могет быть, например, определенный элемент оболочки (любой от первого до последнего), найденной параллельным алгоритмом, должен совпадать с элементом, имеющим тот же номер в оболочке, найденной последовательным алгоритмом. Успешное прохождение всех тестов доказывает корректность работы программы.
\newpage

% Результаты экспериментов
\section*{Результаты экспериментов}
\addcontentsline{toc}{section}{Результаты экспериментов}

\par Эксперименты проводились на векторе разного размера, на 4 процессах (см. Таблица 1).

\par Результаты экспериментов представлены ниже в Таблице 1.

\begin{table}[!h]

\centering
\begin{tabular}{| p{2cm} | p{3cm} | p{4cm} |}
\hline
Размер вектора точек & Время работы последовательного алгоритма (в секундах) & Время работы параллельного алгоритма (в секундах)\\[5pt]
\hline
100       &  0.0000944  &  0.0001555    \\
625       &  0.0001555  &  0.0003325   \\
1225      &  0.0006848  &  0.0013956   \\
2500      &  0.0013131  &  0.0033573  \\
10000     &  0.005069   &  0.0168567    \\
\hline
\end{tabular}
\caption{Результаты вычислительных экспериментов для 4 процессов}
\end{table}


\newpage

% Выводы из результатов экспериментов
\section*{Выводы из результатов экспериментов}
\addcontentsline{toc}{section}{Выводы из результатов экспериментов}
По данным, полученным в результате экспериментов, можно сделать вывод о том, что параллельный алгоритм работает быстрее, чем последовательный. При росте размера вектора можно заметить увеличение ускорения. Это означает, что чем больше размер исходного набора точек, тем более быстро будет работать параллельный алгоритм относительно последовательного. 

\newpage

% Заключение
\section*{Заключение}
\addcontentsline{toc}{section}{Заключение}
Таким образом, в рамках данной лабораторной работы были разработаны последовательный и параллельный алгоритмы для нахождения минимальной выпуклой оболочки компонент бинарного изображения. Одной из главных задач было сделать алгоритм паралллельным. Из нее также вытекает не менее главная задача достижения более быстрого выполенения параллельного алгоритма относительно последовательного. Проведенные тесты доказали, что программа верно вычисляет оболочку. А результаты вычислительных экспериментов доказывают, что поставленные задачи были выполнены.
\newpage

% Литература
\section*{Литература}
\addcontentsline{toc}{section}{Литература}
\begin{enumerate}
\item Гергель В. П., Стронгин Р. Г. Основы параллельных вычислений для многопроцессорных вычислительных систем. – 2003.
\item Минимальная выпуклая оболочка на плоскости. \newline URL: https://grafika.me/node/161
\item learn.microsoft.com:сайт. \newline URL: https://learn.microsoft.com/ru-ru/message-passing-interface/microsoft-mpi.
\end{enumerate} 
\newpage

% Приложение
\section*{Приложение}
\addcontentsline{toc}{section}{Приложение}

convex\_comp\_bin.h
\begin{lstlisting}
// Copyright 2023 Yunin Daniil
#pragma once
#include <vector>
#include <string>

using std::vector;

vector<int> Labeling(const vector<vector<int>> &image, int width, int height);
void PrintImage(const vector<vector<int>> &image, int width, int height);
void PrintImage(const vector<int> &image, int width, int height);
void CreateLabelingImage(int image[], int width, int height);
bool Fill(const vector<vector<int>> &image, vector<vector<int>> *label_image,
    int x, int y, int label, int height, int width);
vector<int> CountComponents(const vector<int> &image, int width, int height);
vector<int> PointsComponents(const vector<int>& components_num_points, const vector<int>& image, int width);
int FindNumComponents(const vector<int> &image);
vector<int> MakeMinPointsLocal(const vector<int> &image, int width, int height, int component);
vector<int> GrahamAlgo(vector<int> points);
vector<int> MainFuncSequence(const vector<int> &image, int width, int height, int num_components);
vector<int> MainFunParallel(const vector<int> &image, int width, int height, int num_components);
int CountNumPointsComponent(vector<int> image);
void CreateComponent1(vector<vector<int>> *matr);
void CreateComponent2(vector<vector<int>> *matr);
void CreateComponent3(vector<vector<int>> *matr, int width, int height);
void CreateComponent4(vector<vector<int>> *matr, int width, int height);
void CreateComponent5(vector<vector<int>> *matr, int width, int height);

\end{lstlisting}
convex\_comp\_bin.cpp
\begin{lstlisting}
// Copyright 2023 Yunin Daniil
#include <mpi.h>
#include <vector>
#include <string>
#include <map>
#include <functional>
#include <random>
#include <algorithm>
#include <iostream>
#include "../../../modules/task_3/yunin_d_convex_comp_bin/convex_comp_bin.h"

using std::vector;

// 1 - Labeling
vector<int> Labeling(const vector<vector<int>> &image, int width, int height) {
    int label = 1;
    vector<vector<int>> labeling_image_arr(height, vector<int>(width));
    // start init array
    for (int i = 0; i < height; i++) {
        for (int j = 0; j < width; j++) {
            labeling_image_arr[i][j] = 0;
        }
    }
    // end init array
    for (int y = 0; y < height; y++) {
        for (int x = 0; x < width; x++) {
            if (Fill(image, &labeling_image_arr, x, y, label, height, width)) {
                label++;
            }
            // std::cout << "\nNext point\n\n";
        }
    }
    // start print array
    // for (int i = 0; i < height; i++) {
    //     for (int j = 0; j < width; j++) {
    //         std::cout << labeling_image_arr[i][j] << ' ';
    //     }
    //     std::cout << std::endl;
    // }
    // end print array
    vector<int> labeling_image(width*height);
    // start init vector
    for (int i = 0; i < height; i++) {
        for (int j = 0; j < width; j++) {
            labeling_image[i * width + j] = labeling_image_arr[i][j];
        }
    }
    // end init vector
    // start print vector
    // std::cout << "print vector";
    // for (int i = 0; i < height * width; i++) {
    //     if ((i % width == 0)) {
    //         std::cout << "\n" << labeling_image[i] << ' ';
    //         continue;
    //     }
    //     std::cout << labeling_image[i] << ' ';
    // }
    // end print vector
    return labeling_image;
}

bool Fill(const vector<vector<int>> &image, vector<vector<int>> *label_image,
    int x, int y, int label, int height, int width) {
    // std::cout << label << std::endl;
    bool next_step = false;
    if (((*label_image)[y][x] == 0) && (image[y][x] == 1)) {
        next_step = true;
        (*label_image)[y][x] = label;
        if (x > 0) {
            // std::cout << "Left\n";
            Fill(image, label_image, x - 1, y, label, height, width);
        }
        if (x < width - 1) {
            // std::cout << "Right\n";
            Fill(image, label_image, x + 1, y, label, height, width);
        }
        if (y > 0) {
            // std::cout << "Down\n";
            Fill(image, label_image, x, y - 1, label, height, width);
        }
        if (y < height - 1) {
            // std::cout << "Up\n";
            Fill(image, label_image, x, y + 1, label, height, width);
        }
    }
    return next_step;
}

void PrintImage(const vector<vector<int>> &image, int width, int height) {
    for (int i = 0; i < height; i++) {
        for (int j = 0; j < width; j++) {
            std::cout << image[i][j] << " ";
        }
        std::cout << std::endl;
    }
}

void PrintImage(const vector<int> &image, int width, int height) {
    for (int i = 0; i < height; i++) {
        for (int j = 0; j < width; j++) {
            std::cout << image[i * width + j] << " ";
        }
        std::cout << std::endl;
    }
}

void CreateLabelingImage(int image[], int width, int height) {
    int default_value = 0;
    for (int i = 0; i < height; i++) {
        for (int j = 0; j < width; i++) {
            image[i * width + j] = 0;
        }
    }
}

vector<int> CountComponents(const vector<int> &image, int width, int height) {
    int num_components = 0;
    for (int i = 0; i < width * height; i++) {
        if (image[i] > num_components) {
            num_components = image[i];
        }
    }
    vector<int> num_points_component(num_components);
    vector<int> components(num_components);
    for (int i = 1; i <= num_components; i++) {
        components[i-1] = i;
    }
    for (int i = 0; i < width * height; i++) {
        for (int j = 0; j < num_components; j++) {
            if (components[j] == image[i]) {
                num_points_component[j]++;
                break;
            }
        }
    }
    vector<int> result(num_components * 2);
    int j = 0;
    for (int i = 0; i < num_components * 2; i++) {
        if (i % 2 == 0) {
            result[i] = components[j];
        } else {
            result[i] = num_points_component[j];
            j++;
        }
    }
    // // start print
    // std::cout << "\n";
    // for (int i = 0; i < num_components * 2; i+=2) {
    //     std::cout << result[i] << " " << result[i + 1] << std::endl;
    // }
    // end print
    return result;
}

int FindNumComponents(const vector<int> &image) {
    int num_components = 0;
    for (int i = 0; i < image.size(); i++) {
        if (image[i] > num_components) {
            num_components = image[i];
        }
    }
    return num_components;
}

vector<int> PointsComponents(const vector<int>& components_num_points, const vector<int>& image, int width) {
    int size = 0;
    for (int i = 1; i < components_num_points.size(); i+=2) {
        size += components_num_points[i];
    }
    size = size * 2;
    size += components_num_points[components_num_points.size() - 2];
    vector<int> points(size);
    std::map<int, vector<int>> map_points;
    for (int i = 0; i < image.size(); i++) {
        for (int j = 0; j < components_num_points.size(); j+=2) {
            if (components_num_points[j] == image[i]) {
                map_points[image[i]].push_back(i);
            }
        }
    }
    // start print
    // for (int i = 1; i <= components_num_points[components_num_points.size() - 2]; i++) {
    //     std::cout << "point " << i << std::endl;
    //     for (int j = 0; j < map_points[i].size(); j++) {
    //         std::cout << map_points[i][j] << " ";
    //     }
    //     std::cout << std::endl;
    // }
    // end print
    int k = 0;
    for (int i = 1; i <= components_num_points[components_num_points.size() - 2]; i++) {
        for (int j = 0; j < map_points[i].size(); j++) {
            points[k] = map_points[i][j] % width;
            k++;
            points[k] = map_points[i][j] / width;
            k++;
        }
        points[k] = -1;
        k++;
    }
    // start print
    // for (int i = 0; i < size; i++) {
    //     std::cout << points[i] << " ";
    // }
    // std::cout << std::endl;
    // end print
    return points;
}
// 1 - Labeling

// 2 - start make min points
int CountNumPointsComponent(vector<int> image) {
    int num_points = 0;
    for (int i = 0; i < image.size(); i++) {
        if (image[i] != 0) {
            num_points++;
        }
    }
    return num_points;
}

vector<int> MakeMinPointsLocal(const vector<int> &image, int width, int height, int component) {
    vector<int> local_image(image);
    for (int i = 0; i < height; i++) {
        for (int j = 0; j < width; j++) {
            if (image[i * width + j] == component) {
                if ((j > 0) && (j < width - 1)) {
                    if ((i == 0) || (i == height - 1)) {
                        if ((image[i * width + j - 1] == component) && (image[i * width + j + 1] == component)) {
                            local_image[i * width + j] = 0;
                        }
                    } else {
                        if (((image[i * width + j - 1] == component) && (image[i * width + j + 1] == component)) ||
                            ((image[(i + 1) * width + j] == component) && (image[(i - 1) * width + j] == component))) {
                            local_image[i * width + j] = 0;
                        }
                    }
                    continue;
                }
                if ((i > 0) && (i < height - 1)) {
                    if ((j == 0) || (j == width - 1)) {
                        if ((image[(i - 1) * width + j] == component) &&
                            (image[(i + 1) * width + j + 1] == component)) {
                            local_image[i * width + j] = 0;
                        }
                    } else {
                        if (((image[i * width + j - 1] == component) && (image[i * width + j + 1] == component)) ||
                            ((image[(i + 1) * width + j] == component) && (image[(i - 1) * width + j] == component))) {
                            local_image[i * width + j] = 0;
                        }
                    }
                }
            } else {
                local_image[i * width + j] = 0;
            }
        }
    }
    // std::cout << "\nPrint for " << component << "\n\n";
    // PrintImage(local_image, width, height);
    int size = CountNumPointsComponent(local_image);
    vector<int> points(size * 2);
    int k = 0;
    for (int i = 0; i < height; i++) {
        for (int j = 0; j < width; j++) {
            if (local_image[i * width + j] != 0) {
                points[k] = j;  // write x coord
                k++;
                points[k] = i;  // write y coord
                k++;
            }
        }
    }
    // print
    // for (int i = 0; i < points.size(); i+=2) {
    //     std::cout << "(" << points[i] << "," << points[i + 1] << ")" << " ";
    // }
    // std::cout << std::endl;
    // end print
    return points;
}
// 2 - end make min points

// 3 start graham algorithm
int rotate(int x1, int y1, int x2, int y2, int x3, int y3) {
    return ((x2 - x1) * (y3 - y2) - (x3 - x2) * (y2 - y1));
}

vector<int> SortingPoints(const vector<int> &points, int x_min, int y_min) {
    vector<int> result(points);
    // print
    // for (int i = 0; i < result.size(); i+=2) {
    //     std::cout << "(" << result[i] << "," << result[i + 1] << ")" << " ";
    // }
    // std::cout << std::endl;
    // end print
    int size = points.size() / 2;
    for (int i = 1; i < size; i++) {
        int j = i;
        while ((j > 0) && (rotate(x_min, y_min, result[2 * j - 2],
            result[2 * j - 1], result[2 * j], result[2 * j + 1]) < 0)) {
            int temp = result[2 * j - 2];
            result[2 * j - 2] = result[2 * j];
            result[2 * j] = temp;
            temp = result[2 * j - 1];
            result[2 * j - 1] = result[2 * j + 1];
            result[2 * j + 1] = temp;
            j--;
        }
    }
    // print
    // for (int i = 0; i < result.size(); i+=2) {
    //     std::cout << "(" << result[i] << "," << result[i + 1] << ")" << " ";
    // }
    // std::cout << std::endl;
    // end print
    return result;
}

vector<int> GrahamAlgo(vector<int> points) {
    vector<int> result;
    int num_points = points.size() / 2;
    if (num_points > 1) {
        int x_min = points[0];
        int y_min = points[1];
        int min_index = 0;
        for (int i = 2; i < points.size(); i+=2) {
            if (points[i] < x_min || (points[i] == x_min && points[i + 1] < y_min)) {
                x_min = points[i];
                y_min = points[i + 1];
                min_index = i;
            }
        }
        // delete first point from vector
        int temp = points[min_index];
        points[min_index] = points[num_points * 2 - 2];
        points[num_points * 2 - 2] = temp;
        temp = points[min_index + 1];
        points[min_index + 1] = points[num_points * 2 - 1];
        points[num_points * 2 - 1] = temp;
        points.pop_back();
        points.pop_back();
        // std::cout << "\nxMin = " << x_min << ", yMin = " << y_min << '\n';
        points = SortingPoints(points, x_min, y_min);
        // start print
        // for (int i = 0; i < points.size(); i+=2) {
        //     std::cout << "(" << points[i] << "," << points[i + 1] << ")" << " ";
        // }
        // std::cout << std::endl;
        // end print
        result.push_back(x_min);
        result.push_back(y_min);
        result.push_back(points[0]);
        result.push_back(points[1]);
        for (int i = 2; i < points.size(); i += 2) {
            int result_size = result.size();
            // p1
            int x1 = result[result_size - 4];
            int y1 = result[result_size - 3];
            // p2
            int x2 = result[result_size - 2];
            int y2 = result[result_size - 1];
            // p3
            int x3 = points[i];
            int y3 = points[i + 1];

            int rot = rotate(x1, y1, x2, y2, x3, y3);
            // p1, p2, p3 on one line -> replacing p2 with p3
            if (rot == 0) {
                result[result_size - 2] = x3;
                result[result_size - 1] = y3;
            } else if (rot < 0) {
                // While p1, p2, p3 form right turn -> excluding p2
                while (rotate(result[(result.size()) - 4], result[(result.size()) - 3],
                            result[(result.size()) - 2], result[(result.size()) - 1], x3, y3) < 0)
                    result.pop_back(), result.pop_back();
                result.push_back(x3);
                result.push_back(y3);
            } else {
                // left turns are ok - just adding p3 to our result
                result.push_back(x3);
                result.push_back(y3);
            }
        }
        // start print
        // for (int i = 0; i < result.size(); i+=2) {
        //     std::cout << "(" << result[i] << "," << result[i + 1] << ")" << " ";
        // }
        // std::cout << std::endl;
        // end print
    } else {
        result.resize(2);
        result[0] = points[0];
        result[1] = points[1];
    }
    return result;
}
// 3 end graham algorithm

// start 4 components
void CreateComponent1(vector<vector<int>> *matr) {
    (*matr)[0][0] = 1; (*matr)[2][3] = 1; (*matr)[1][3] = 1;
    (*matr)[2][0] = 1; (*matr)[0][2] = 1; (*matr)[2][2] = 1;
    (*matr)[1][1] = 1; (*matr)[0][1] = 1; (*matr)[2][1] = 1;
    (*matr)[9][9] = 1; (*matr)[9][8] = 1; (*matr)[9][7] = 1;
    (*matr)[9][6] = 1; (*matr)[9][5] = 1; (*matr)[9][4] = 1;
    (*matr)[9][1] = 1; (*matr)[4][4] = 1; (*matr)[4][5] = 1;
    (*matr)[4][3] = 1; (*matr)[4][6] = 1; (*matr)[3][6] = 1;
    (*matr)[8][2] = 1; (*matr)[7][8] = 1; (*matr)[6][2] = 1;
}

void CreateComponent2(vector<vector<int>> *matr) {
    (*matr)[1][1] = 1;
    (*matr)[2][1] = 1;
    (*matr)[3][1] = 1;
}

void CreateComponent3(vector<vector<int>> *matr, int width, int height) {
    (*matr)[1][1] = 1; (*matr)[2][1] = 1; (*matr)[3][1] = 1;
    (*matr)[1][2] = 1; (*matr)[2][2] = 1; (*matr)[3][2] = 1;
    (*matr)[1][3] = 1; (*matr)[2][3] = 1; (*matr)[3][3] = 1;
    (*matr)[1][11] = 1; (*matr)[2][11] = 1; (*matr)[3][11] = 1;
    (*matr)[1][12] = 1; (*matr)[2][12] = 1; (*matr)[3][12] = 1;
    (*matr)[1][13] = 1; (*matr)[2][13] = 1; (*matr)[3][13] = 1;
    (*matr)[11][1] = 1; (*matr)[12][1] = 1; (*matr)[13][1] = 1;
    (*matr)[11][2] = 1; (*matr)[12][2] = 1; (*matr)[13][2] = 1;
    (*matr)[11][3] = 1; (*matr)[12][3] = 1; (*matr)[13][3] = 1;
    (*matr)[11][11] = 1; (*matr)[12][11] = 1; (*matr)[13][11] = 1;
    (*matr)[11][12] = 1; (*matr)[12][12] = 1; (*matr)[13][12] = 1;
    (*matr)[11][13] = 1; (*matr)[12][13] = 1; (*matr)[13][13] = 1;
}



void CreateComponent4(vector<vector<int>> *matr, int width, int height) {
    for (int i = 0; i < height; i++) {
        if (i % 2 == 0) {
           for (int j = 0; j < width; j++) {
               (*matr)[i][j] = 1;
            }
        } else {
            for (int j = 1; j < width; j+=2) {
                (*matr)[i][j] = 1;
            }
        }
    }
}

void CreateComponent5(vector<vector<int>> *matr, int width, int height) {
    for (int i = 0; i < height; i++) {
        if (i == 0 || i == height - 1) {
           for (int j = 0; j < width; j++) {
               (*matr)[i][j] = 1;
            }
        } else {
            (*matr)[i][0] = 1;
            (*matr)[i][width - 1] = 1;
        }
    }
    (*matr)[2][2] = 1; (*matr)[1][1] = 1; (*matr)[3][3] = 1;
    (*matr)[3][1] = 1; (*matr)[1][3] = 1;
}
// end 4 components

vector<int> MainFuncSequence(const vector<int> &image, int width, int height, int num_components) {
    vector<int> result;
    for (int i = 1; i <= num_components; i++) {
        vector<int> local_image(image);
        vector<int> points_component = MakeMinPointsLocal(local_image, width, height, i);
        vector<int> convex_shell = GrahamAlgo(points_component);
        // start print
        // std::cout << "convex shell" <<std::endl;
        // for (int i = 0; i < convex_shell.size(); i+=2) {
        //     std::cout << "(" << convex_shell[i] << "," << convex_shell[i + 1] << ")" << " ";
        // }
        // std::cout << std::endl;
        // end print
        for (int j = 0; j < convex_shell.size(); j++) {
            result.push_back(convex_shell[j]);
        }
        result.push_back(-1);
    }
    // start print
    // std::cout << "result" <<std::endl;
    // for (int i = 0; i < result.size() - 2; i+=2) {
    //     if (result[i] == -1) {
    //         std::cout << result[i] << " ";
    //         i++;
    //     }
    //     std::cout << "(" << result[i] << "," << result[i + 1] << ")" << " ";
    // }
    // std::cout << std::endl;
    // end print
    return result;
}


vector<int> MainFunParallel(const vector<int> &image, int width, int height, int num_components) {
    vector<int> result;
    int comm_size, rank;
    MPI_Comm_size(MPI_COMM_WORLD, &comm_size);
    MPI_Comm_rank(MPI_COMM_WORLD, &rank);
    // std::cout << "all proc " << comm_size << " rank " << rank << "\n";
    const int num_active_proc = std::min(comm_size, num_components);
    // std::cout << "num active proc " << num_active_proc << "\n";
    // std::cout << "num compoments " << num_components << "\n";
    if (rank >= num_active_proc) {
        return vector<int>(0);
    }
    vector<int> components(num_components);
    for (int i = 0; i < num_components; i++) {
        components[i] = i+1;
    }
    int part = num_components / num_active_proc;
    int remainder = num_components % num_active_proc;
    int local_parts = part;
    if (rank < remainder) {
        local_parts++;
    }
    vector<int> local_components(local_parts);
    vector<int> local_image(width * height);
    if (rank == 0) {
        local_image = image;
        for (int i = 1; i < num_active_proc; i++) {
            // std::cout << "Send 1\n";
            MPI_Send(local_image.data(), width * height, MPI_INT, i, 0, MPI_COMM_WORLD);
        }
    } else {
        MPI_Status stat;
        // std::cout << "Recv 1\n";
        MPI_Recv(local_image.data(), width * height, MPI_INT, 0, 0, MPI_COMM_WORLD, &stat);
    }
    if (rank != 0) {
        MPI_Send(&local_parts, 1, MPI_INT, 0, 0, MPI_COMM_WORLD);
    }
    if (rank == 0) {
        int index = 0;
        for (int i = 1; i < num_active_proc; i++) {
            // index = (i-1) * part;
            int proc_part;
            MPI_Status stat;
            MPI_Recv(&proc_part, 1, MPI_INT, i, 0, MPI_COMM_WORLD, &stat);
            // std::cout << "rank = " << rank << " index = " << index << " proc_part " << proc_part << "\n";
            // std::cout << "Send 2\n";
            // std::cout << " rank = " << i << " part = " << proc_part << "\n";
            MPI_Send(components.data() + index, proc_part, MPI_INT, i, 0, MPI_COMM_WORLD);
            index += proc_part;
        }
        // index += part;
        // if (num_active_proc == 1) {
        //     index--;
        // }
        // std::cout << "rank = " << rank << " index = " << index << " part " << local_parts << "\n";
        local_components.resize(local_parts);
        int j = 0;
        for (int i = index; i < num_components; i++) {
            local_components[j] = components[i];
            j++;
        }
    } else {
        MPI_Status stat;
        // std::cout << "Recv 1\n";
        MPI_Recv(local_components.data(), local_parts, MPI_INT, 0, 0, MPI_COMM_WORLD, &stat);
    }
    // std::cout << "Proc " << rank << "\n";
    // PrintImage(local_image, width, height);
    // for (int i = 0; i < local_components.size(); i++) {
    //     std::cout << local_components[i] << ' ';
    // }
    // std::cout << "\n";
    // Same operations for process
    vector<int> local_result;
    for (int i = 0; i < local_components.size(); i++) {
        vector<int> points_component = MakeMinPointsLocal(local_image, width, height, local_components[i]);
        vector<int> convex_shell = GrahamAlgo(points_component);
        for (int j = 0; j < convex_shell.size(); j++) {
            local_result.push_back(convex_shell[j]);
        }
        local_result.push_back(-1);
    }
    // print
    // std::cout << "Proc " << rank << "\n";
    // for (int i = 0; i < local_result.size(); i++) {
    //     std::cout << local_result[i] << ' ';
    // }
    // std::cout << "\n";
    // print
    if (rank != 0) {
        // std::cout << "Send 3\n";
        MPI_Send(local_result.data(), local_result.size(), MPI_INT, 0, 0, MPI_COMM_WORLD);
    } else {
        int size = 2 * CountNumPointsComponent(local_image);
        size += FindNumComponents(local_image);
        // std::cout << "recv 3\n";
        for (int i = 1; i < num_active_proc; i++) {
            vector<int> buffer(size);
            MPI_Status stat;
            MPI_Recv(buffer.data(), size, MPI_INT, i, 0, MPI_COMM_WORLD, &stat);
            // print
            // std::cout << "\nRank = " << i << std::endl;
            // for (int i = 0; i < size; i++) {
            //     if (buffer[i] == -1 && buffer[i+1] == 0) {
            //         std::cout << buffer[i] << " ";
            //         break;
            //     }
            //     std::cout << buffer[i] << " ";
            // }
            // std::cout << "\n";
            // print
            for (int i = 0; i < size; i++) {
                if (buffer[i] == -1 && buffer[i+1] == 0) {
                    result.push_back(buffer[i]);
                    break;
                }
                result.push_back(buffer[i]);
            }
        }
        // print
        // for (int i = 0; i < local_result.size(); i++) {
        //     std::cout << local_result[i] << " ";
        // }
        // std::cout << "\n";
        // print
        for (int i = 0; i < local_result.size(); i++) {
            result.push_back(local_result[i]);
        }
    }
    return result;
}

\end{lstlisting}
main.cpp
\begin{lstlisting}
// Copyright 2023 Yunin Daniil
#include <mpi.h>
#include <gtest/gtest.h>
#include <vector>
#include <gtest-mpi-listener.hpp>
#include "./convex_comp_bin.h"

TEST(Convex_Shell_Binary_Image, Test_1) {
    int rank;
    MPI_Comm_rank(MPI_COMM_WORLD, &rank);
    int width = 10, height = 10;
    std::vector<std::vector<int>> global_vec(height, std::vector<int>(width));
    std::vector<int> labeling_image(width*height);
    // std::vector<int> global_vec(7*7);
    // if (rank == 0) {
    //     CreateComponent1(&global_vec);
    //     PrintImage(global_vec, width, height);
    // }
    CreateComponent1(&global_vec);
    labeling_image = Labeling(global_vec, width, height);
    if (rank == 0) {
        // PrintImage(labeling_image, width, height);
    }
    vector<int> convex_shell_par = MainFunParallel(labeling_image, width, height, FindNumComponents(labeling_image));
    if (rank == 0) {
        vector<int> convex_shell_seq = MainFuncSequence(labeling_image,
            width, height, FindNumComponents(labeling_image));
        // std::cout << "Checked\n";
        // std::cout << "Seq " << convex_shell_seq.size() << " Par " <<  convex_shell_par.size() << "\n";
        // for (int i = 0; i < convex_shell_seq.size(); i++) {
        //     std::cout << "Value seq = " << convex_shell_seq[i] << " Value par " << convex_shell_par[i] << "\n";
        // }
        for (int i = 0; i < convex_shell_seq.size(); i++) {
            ASSERT_EQ(convex_shell_seq[i], convex_shell_par[i]);
        }
    }
}

TEST(Convex_Shell_Binary_Image, Test_2) {
    int rank;
    MPI_Comm_rank(MPI_COMM_WORLD, &rank);
    int width = 4, height = 4;
    std::vector<std::vector<int>> global_vec(height, std::vector<int>(width));
    std::vector<int> labeling_image(width*height);
    // std::vector<int> global_vec(7*7);
    // if (rank == 0) {
    //     CreateComponent1(&global_vec);
    //     PrintImage(global_vec, width, height);
    // }
    CreateComponent2(&global_vec);
    // PrintImage(global_vec, width, height);
    labeling_image = Labeling(global_vec, width, height);
    // PrintImage(labeling_image, width, height);
    vector<int> convex_shell_par = MainFunParallel(labeling_image, width, height, FindNumComponents(labeling_image));
    if (rank == 0) {
        vector<int> convex_shell_seq = MainFuncSequence(labeling_image,
            width, height, FindNumComponents(labeling_image));
        // std::cout << "Checked\n";
        // std::cout << "Seq " << convex_shell_seq.size() << " Par " <<  convex_shell_par.size() << "\n";
        // for (int i = 0; i < convex_shell_seq.size(); i++) {
        //     std::cout << "Value seq = " << convex_shell_seq[i] << " Value par " << convex_shell_par[i] << "\n";
        // }
        for (int i = 0; i < convex_shell_seq.size(); i++) {
            ASSERT_EQ(convex_shell_seq[i], convex_shell_par[i]);
        }
    }
}

TEST(Convex_Shell_Binary_Image, Test_3) {
    int rank;
    MPI_Comm_rank(MPI_COMM_WORLD, &rank);
    int width = 15, height = 15;
    std::vector<std::vector<int>> global_vec(height, std::vector<int>(width));
    std::vector<int> labeling_image(width*height);
    // std::vector<int> global_vec(7*7);
    // if (rank == 0) {
    //     CreateComponent1(&global_vec);
    //     PrintImage(global_vec, width, height);
    // }
    CreateComponent3(&global_vec, width, height);
    // PrintImage(global_vec, width, height);
    labeling_image = Labeling(global_vec, width, height);
    // PrintImage(labeling_image, width, height);
    vector<int> convex_shell_par = MainFunParallel(labeling_image, width, height, FindNumComponents(labeling_image));
    if (rank == 0) {
        vector<int> convex_shell_seq = MainFuncSequence(labeling_image,
            width, height, FindNumComponents(labeling_image));
        // std::cout << "Checked\n";
        // std::cout << "Seq " << convex_shell_seq.size() << " Par " <<  convex_shell_par.size() << "\n";
        // for (int i = 0; i < convex_shell_seq.size(); i++) {
        //     std::cout << "Value seq = " << convex_shell_seq[i] << " Value par " << convex_shell_par[i] << "\n";
        // }
        for (int i = 0; i < convex_shell_seq.size(); i++) {
            ASSERT_EQ(convex_shell_seq[i], convex_shell_par[i]);
        }
    }
}

TEST(Convex_Shell_Binary_Image, Test_4) {
    int rank;
    MPI_Comm_rank(MPI_COMM_WORLD, &rank);
    int width = 7, height = 7;
    std::vector<std::vector<int>> global_vec(height, std::vector<int>(width));
    std::vector<int> labeling_image(width*height);
    // std::vector<int> global_vec(7*7);
    // if (rank == 0) {
    //     CreateComponent1(&global_vec);
    //     PrintImage(global_vec, width, height);
    // }
    CreateComponent4(&global_vec, width, height);
    // PrintImage(global_vec, width, height);
    // std::cout << "\n";
    labeling_image = Labeling(global_vec, width, height);
    // PrintImage(labeling_image, width, height);
    vector<int> convex_shell_par = MainFunParallel(labeling_image, width, height, FindNumComponents(labeling_image));
    if (rank == 0) {
        vector<int> convex_shell_seq = MainFuncSequence(labeling_image,
            width, height, FindNumComponents(labeling_image));
        // std::cout << "Checked\n";
        // std::cout << "Seq " << convex_shell_seq.size() << " Par " <<  convex_shell_par.size() << "\n";
        // for (int i = 0; i < convex_shell_seq.size(); i++) {
        //     std::cout << "Value seq = " << convex_shell_seq[i] << " Value par " << convex_shell_par[i] << "\n";
        // }
        for (int i = 0; i < convex_shell_seq.size(); i++) {
            ASSERT_EQ(convex_shell_seq[i], convex_shell_par[i]);
        }
    }
}

TEST(Convex_Shell_Binary_Image, Test_5) {
    int rank;
    MPI_Comm_rank(MPI_COMM_WORLD, &rank);
    int width = 5, height = 5;
    std::vector<std::vector<int>> global_vec(height, std::vector<int>(width));
    std::vector<int> labeling_image(width*height);
    // std::vector<int> global_vec(7*7);
    // if (rank == 0) {
    //     CreateComponent1(&global_vec);
    //     PrintImage(global_vec, width, height);
    // }
    CreateComponent5(&global_vec, width, height);
    // PrintImage(global_vec, width, height);
    // std::cout << "\n";
    labeling_image = Labeling(global_vec, width, height);
    // PrintImage(labeling_image, width, height);
    vector<int> convex_shell_par = MainFunParallel(labeling_image, width, height, FindNumComponents(labeling_image));
    if (rank == 0) {
        vector<int> convex_shell_seq = MainFuncSequence(labeling_image,
            width, height, FindNumComponents(labeling_image));
        // std::cout << "Checked\n";
        // std::cout << "Seq " << convex_shell_seq.size() << " Par " <<  convex_shell_par.size() << "\n";
        // for (int i = 0; i < convex_shell_seq.size(); i++) {
        //     std::cout << "Value seq = " << convex_shell_seq[i] << " Value par " << convex_shell_par[i] << "\n";
        // }
        for (int i = 0; i < convex_shell_seq.size(); i++) {
            ASSERT_EQ(convex_shell_seq[i], convex_shell_par[i]);
        }
    }
}

int main(int argc, char** argv) {
    ::testing::InitGoogleTest(&argc, argv);
    MPI_Init(&argc, &argv);

    ::testing::AddGlobalTestEnvironment(new GTestMPIListener::MPIEnvironment);
    ::testing::TestEventListeners& listeners =
        ::testing::UnitTest::GetInstance()->listeners();

    listeners.Release(listeners.default_result_printer());
    listeners.Release(listeners.default_xml_generator());

    listeners.Append(new GTestMPIListener::MPIMinimalistPrinter);
    return RUN_ALL_TESTS();
}


\end{lstlisting}

\end{document}